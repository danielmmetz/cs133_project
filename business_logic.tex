\documentclass{article}

\begin{document}

\noindent Below are the key aspects of the business logic that are integral to the functionality of the RoomDraw application.

\begin{itemize}
\item A student who is a representative of a group makes the request on behalf of the group. The representative of a group is by default the student in the group with the highest $drawNum$.

\item On creation of a group, the $G.drawNum$ is calculated. The manner by which it is calculated depends on the type of a group.  If the group $G$ is to draw into a Friendship suite with students $S_1 \cdots S_r$, then $G.drawNum$ is the arithmetic mean of $S_1.drawNum \cdots S_r.drawNum$. Otherwise, $G.drawNum$ is $S_i.drawNum$ where $S_i$ is a student such that $S_i > S_j$ for all $j \in [r]$.

\item For a tuple $o \in Occupy$, $o.inFall$ is true or $o.inSpring$ is true.

\item Queries for Suites are a search for Collections with non-NULL $suiteNum$ attributes.

\item In a Student's preferences list, the rooms selected by the friendship automatically have higher $absrank$ than the rest.
\end{itemize}

\end{document}
