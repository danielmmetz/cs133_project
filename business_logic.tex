\documentclass{article}

\begin{document}

\noindent Below are the key aspects of the business logic that are integral to the functionality of the RoomDraw application.

\begin{itemize}
\item A student who is a representative of a group makes the request on behalf of the group. The representative of a group is by default the student in the group with the highest $drawNum$.

\item On creation of a group, the $G.drawNum$ is calculated. The manner by which it is calculated depends on the type of a group.  If the group $G$ is to draw into a Friendship suite with students $S_1 \cdots S_r$, then $G.drawNum$ is the arithmetic mean of $S_1.drawNum \cdots S_r.drawNum$. Otherwise, $G.drawNum$ is $S_i.drawNum$ where $S_i$ is a student such that $S_i > S_j$ for all $j \in [r]$.

\item For a tuple $o \in Occupy$, $o.inFall$ is true or $o.inSpring$ is true.

\item Queries for Suites are a search for Collections with non-NULL $suiteNum$ attributes.

\item On the draw page, Collections are added to a students preferences list.  then, on the preferences page, when they press \textit{Submit}, the corresponding requests are saved to the database.

\item The number of students in any group must be between 1 and 6 (inclusive).

\item The attribute $absRank$ in the $Request$ relation is a \texttt{real} to facilitate the ordering of requests.  We set the maxmium number of requests to be 100, then the first two decimal places specify the student's ordered preference. In other words, if a representative student $s$ specifies an ordering $1,2,\cdots, n$, then the requests $r_1,r_2, \cdots, r_n \in Request$ are created such that \[r_i.absRank = s.drawNum + i/100.0\]

\item The minimum $drawNum$ for students is $2000$. This allows for friendship suites to be given priority. On commiting the requests $r_1\cdots r_n$, if $r_i.cid$ corresponds to a friendship suite, $r_i.drawNum$ is automatically reduced by this minimum $drawNum$, $2000$. This gives requests for a friendship suite preferred $absRank$ values.

\end{itemize}

\end{document}
