This section details some of the difficulties we've encountered while working on
this project. Some difficulties have been conquered, others have been relegated
to future extensions we'd like to later support.

\subsection{Challenges Conquered}

For the most part, our project progressed smoothly. We spent most of our time
not attempting to fix bugs but rather learning Rails and figuring out the best
means through which to implement our desired feature set. Below you'll find some
of the challenges we're particularly proud to have conquered.

\begin{outline}
\1 Learning Ruby, Rails, Git, and their best practices.
  \2 includes idiomatic ruby, git branching, and merges by pull request
\1 Discovering (often in advance) many edge cases. This includes ensuring
  various invariants like:
  \2 deleting associated requests when a draw group's size changes
  \2 ensuring a student cannot belong to too many groups, including when they
  are added by another student to an existing group
  \2 groups may only add to their queue collections of matching capacity
\1 Maintaining what we believe to be an intuitive and relatively simple model
  for what we slowly realized to be a surprisingly complicated system and
  process.
\1 Ensuring reasonable load times on all pages.
  \2 Of particular note is the search page, which efficiently displays 700+
  collections for various queries. The initial implementation resulted in a 24
  second load time for the page. The current implementation reduces this load
  time below 1 second.
\1 One-upping the current room draw process to incorporate automation.
  \2 This is achieved by use of the preference queue and the \texttt{draw} rake
  task.
\end{outline}

\subsection{Challenges To Conquer}

While there's much we're proud to have achieved, there's still much we'd like to
incorporate to extend our project. Some of these extensions are listed out
below:

\begin{outline}
\1 General Extensions:
  \2 Better handle students abroad and/or on leave.
    \3 Currently we handle this by simply assuming that such students will have
    no collections in their queue. We have some initial hooks to handle
    this---on each student we have a boolean \texttt{is\_absent} that aims
    to indicate such information. We would need only to prevent absent students
    from being a part of any draw group.
  \2 Display and retreive historical data.
    \3 Allow students to see the historical group draw numbers for each
    collection. This allows students to gauge their likelihood to draw into a
    collection in question and mirrors a feature of the current real-life
    system.
  \2 Generate draw numbers.
    \3 Given historical data (above), we can generate the draw numbers for
    students based on their past tiers placements.
  \2 Prevent suite-size group creations and modificiations at some desired time.
    \3 If we wish to more closely resemble the current Room Draw system, we can
    incorporate this. This can be handled by simply checking against a variable
    to indicate when this condition is to be considered, and checking against
    the current group size before adding another member.
  \2 Hide draw numbers at some desired time.
    \3 Similar to the above.
  \2 Better session tracking (BUGFIX)
    \3 Currently, if a logged-in user manually routes to the \texttt{/login}
    page, the site routing breaks. This should be fixable by rerouting the
    \texttt{/login} page.
  \2 Modify the rooms and collections to match the current set of rooms and
  collections set by Pomona College.
  \2 Room assignment.
    \3 After a group has successfully drawn into a collection, allow group
    members to assign themselves to rooms.

\1 Admin Pages
  \2 Create a set of pages to enable admin control over:
    \3 Collections and their associated Rooms
    \3 Student accounts
    \3 The Occupy relation (e.g. room-change requests)
    \3 A Student-to-Room Relation (when it exists)

\1 AutoDraw:
  \2 Map draw numbers to draw times.
    \3 This would allow it so that a student can know when to check their
    assignment status.
    \3 It would also allow students to know when to check-in, allowing students
    a time to see if they've been assigned a room and to know when to add more
    collections to their queue if necessary. Currently a student whose draw
    groups have only unfulfillable requests (all have been taken by earlier draw
    numbers) is simply passed over. This is not ideal behavior. Ideally, this
    would handle a bit more like course registration.
  \2 Tie-breaking (BUGFIX)
    \3 Consider two draw groups with equal draw number averages, but with member
    draw numbers respectively 1, 3, 5 and 2, 3, 4. Currently if either of these
    groups makes a request, there exists some sort of a ``race condition'' in
    that the request priority of these two groups is interleaved in order of
    their preference creation. In theory, to better model the current process,
    there should exist a tie-breaking process to assign one group higher
    priority over the other. This has been relegated as a low-priority fix given
    the low likelihood of multiple draw groups with identical averages.

\1 UI:
  \2 Better match original wireframe mockups.
    \3 We believe this to be a nicer aesthetic in at least a few regards.
  \2 Better format search results.
    \3 This might include for instance, not listing the rooms of a collection as
    \sbr{a, b, c, d} but instead as \(a, b, c, d\).

\1 Groups:
  \2 Allow deletion of members from a group.
    \3 This may also permit an individual to remove themselves from a group,
    with an option to disband the entire group.
  \2 Require consent to be added to a group.
    \3 This might involve an invitation to invite a group with the option to
    accept or reject the invitation.
  \2 Remove the idea of a group representative.
    \3 This may simply make the various interfaces more intuitive.
    \3 This may also make the system less troublesome as the representative
    cannot currently be elected but is instead automatically assigned.
  \2 Do not display the \texttt{add} field and button for a group at max
  capacity.

\1 Search:
  \2 Allow an enqueue operation to add to the top of the queue as well as the
  bottom of the queue.
  \2 Extend the search options.
    \3 by room type (function of capacity and number of rooms) (e.g. 2RD vs 1RD)
    \3 by floor
  \2 Lazy loading pagination.
  \2 Coloring of the enqueue buttom.
    \3 This can include indications that a search result is already present in
    your queue

\1 Queue:
  \2 Up/Down arrows on requests.
    \3 This would allow for a simple re-ordering of requests.
  \2 Draggable requests.
    \3 This would allow for a nicer interface interaction for request
    re-ordering.
\end{outline}