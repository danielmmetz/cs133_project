This chapter details the high-levels of our project, spanning its motivation,
current feature-set, and suggestions for future improvements.

\subsection{Motivation}

TODO: discuss how room draw currently works

\subsection{User Manual}

\begin{outline}
\1 Login Page
  \2 The login-page prompts a user to log-in with a student id and password.
  Currently the set of user ids and passwords are independent of those used by
  Pomona College. Note however the system does consider security, and stores
  only password digests encrypted via \texttt{bcrypt}.
  \2 Before Room Draw:
    \3 Upon successful log-in, a student is notified that they do not yet have
    an assigned room. They are also provided with simple navigational
    instructions.
  \2 After Room Draw:
    \3 If a student belongs to a draw group that has successfully drawn into a
    collection, the student is brought to a landing page informing the student
    of such. This provides information as for the students with whom they will
    be living, and the rooms that are collectively assigned to the said group of
    students. GUI access to other features of the site is disabled.
    \3 Otherwise, if a student does not yet belong to a draw group successfully
    assigned to a collection, the web-app behaves as Room Draw has not yet been
    drawn.
\1 Group Management
  \2 By default, all students belong to a group containing only themselves.
  \2 All students may create additional draw groups (up to a max of 50). Doing
  so will create a new draw group, initialized containing the logged-in student.
  \2 Any student belonging to a draw group may add any other student to that
  group via student id. Draw groups are capped at size 6. Draw groups containing
  between 3 and 6 members are considered friendship groups, and have their draw
  numbers calculated accordingly.
  \2 Any student belonging to a group may delete it, which removes all data
  associated with that group.
\1 Collection Search
  \2 The search page supports search by dorm name and by collection capacity.
  For each matching collection, the collection id, dorm name, room number(s),
  and room capacity(-ies) are listed. Group representatives may select a group
  they represent and add it to that group's preferences queue.
\1 Preference Queue
  \2 The Queue page lists out the various requests made by groups to which the
  logged-in student belongs. These requests are sorted first by the draw numbers
  of the groups, then by their addition time to the queue. We offer a small
  example to illustrate our intent: Consider that I add to my queue collection
  \(a\) for my solo group. I then add to my queue collection \(b\) for a
  friendship group I represent. I lastly add collection \(c\) for my solo group.
  The queue will display results in order \(b, a, c\), mirroring the order by
  which the requests would be evaulated at the moment of RoomDraw.
  \2 Requests in the queue may be deleted.
\end{outline}

\subsection{Future Ambitions}



\subsection{Functionalities}

Our idea is a proof of concept of a working Room Draw replacement system. Our
core functionality includes:

\begin{outline}
\1 students signing up for singles, doubles, or friendship suites
\1 students creating draw groups
    \2 groups have a representative who controls group preferences
\1 students searching for rooms
    \2 offer filtering by property (e.g. taken, vacant, location, dorm, sq-ft,
    etc.)
\1 an ``auto draw'' system
    \2 We define an ``auto draw'' system as one in which students may provide an
    ordered list of room preferences, to allow for automated room selection at
    draw time given room availabilities.
\1 including all the actual rooms around campus
\end{outline}

Possible extensions include:

\begin{outline}
\1 include access to room reviews in system
\1 include a notification system
\1 include historical data as context for each room (another property of the
    room)
\1 permit 5C campus exchanges (e.g. CMC student living at Pomona)
\1 generation of room draw numbers/times
\1 administrator override (e.g. allow dean of housing to move students around or modify
    draw numbers/times).
\end{outline}