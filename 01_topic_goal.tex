The existing Room Draw process is profoundly inefficient, requiring students to spend stressful hours
watching their desired rooms slowly disappear, and requiring administrators and RAs to waste time
overseeing every step of the process.  If a student is abroad or for any reason not present at
the time of Room Draw, they must specify another student to be their proxy, and draw for them.
The current system requires two days to run to completion: the first is to draw Friendship Suites 
(groups of three or more students who wish to live together), the second is for general draw in 
which singles and doubles are allocated.  The allocation scheme is different for these two draw 
schedules. Friendship groups are given a draw number equal to the average draw numbers 
of the primary four members, where doubles and singles take the highest draw number 
of their members. Rooms are then allocated from lowest to highest draw number.

This project solves much of this issue by not requiring students to be physically
present to draw into a room, and removes much of the administrative overhead.
Students give a list of their room preferences before some deadline, and then
the administrator runs an automatic room draw task that assigns students to
their rooms based on these preferences. 
