This chapter details the high-levels of our project, spanning its motivation,
current feature-set, and suggestions for future improvements.

\subsection{Motivation}

TODO: discuss how room draw currently works

\subsection{Future Ambitions}



\subsection{Functionalities}

Our idea is a proof of concept of a working Room Draw replacement system. Our
core functionality includes:

\begin{outline}
\1 students signing up for singles, doubles, or friendship suites
\1 students creating draw groups
    \2 groups have a representative who controls group preferences
\1 students searching for rooms
    \2 offer filtering by property (e.g. taken, vacant, location, dorm, sq-ft,
    etc.)
\1 an ``auto draw'' system
    \2 We define an ``auto draw'' system as one in which students may provide an
    ordered list of room preferences, to allow for automated room selection at
    draw time given room availabilities.
\1 including all the actual rooms around campus
\end{outline}

Possible extensions include:

\begin{outline}
\1 include access to room reviews in system
\1 include a notification system
\1 include historical data as context for each room (another property of the
    room)
\1 permit 5C campus exchanges (e.g. CMC student living at Pomona)
\1 generation of room draw numbers/times
\1 administrator override (e.g. allow dean of housing to move students around or modify
    draw numbers/times).
\end{outline}