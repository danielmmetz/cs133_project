\documentclass{article}

\begin{document}
\begin{center} \Large Room Draw ER Diagram \end{center}

We model our ER diagram after Pomona College's current room draw system with a few modifications. A \textbf{Student} is required to be a part of at least one \textbf{Draw Group}, which consists of one or more students. We decided to give draw groups a draw number attribute to distinguish between normal draw and suite draw, the latter averaging individual student draw numbers. Furthermore, each draw group must have a \textbf{Representative} who will select a \textbf{Collection} of one or more rooms for the group, depending on the size of the group. All students are representatives of themselves in a group containing only themselves. Similar to the student/draw group coupling, a \textbf{Room} is required to be a part of exactly one collection. These collections represent individual rooms and multiple associated rooms such as two-room doubles and friendship suites. A representative may \textbf{Request} a collection and rank the request for automated room draw. That is, if any requested collections are available at the representative's draw time, the highest ranked collection will be drawn for the group. These draw times correspond to a group's absolute rank attribute which is determined by whether or not the group is a friendship suite and by the group's draw number. Just as a note, not every draw group will make a request as groups containing students who are studying abroad will not participate in the request relationship. Along the same vein, not all rooms on Pomona's campus are occupied at any given time so not all collections will participate in the request relationship. For the same reasons, the \textbf{Occupy} relationship does not require total participation by either students or rooms. 

\end{document}
