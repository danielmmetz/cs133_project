The result of the translation of the ER diagram to the schema is presented in
the next section, as well as the explanation of any non-obvious choices. First,
consider the \emph{Student} and \emph{DrawGroup} entity sets and their mutual
relationships.

\begin{description}
  \item Student(
        \ul{\emph{student_id}: \texttt{integer}},
        \emph{password}: \texttt{string},
        \emph{draw\_num}: \texttt{integer},
        \emph{name}: \texttt{string},
        \emph{grad\_year}: \texttt{integer})

  \item Member(
        \ul{\emph{student\_id}: \texttt{integer},
        \emph{draw\_group\_id}: \texttt{integer}})

  \item DrawGroup(
        \ul{\emph{draw\_group\_id}: \texttt{integer}},
        \emph{draw\_num}: \texttt{integer},
        \emph{student\_id}: \texttt{integer},
        \emph{for\_suite}: \texttt{boolean})
\end{description}

The above schema mostly sticks to the obvious translation.  However, we have key
and total participation constraints on the \emph{Representative} relationship
with \emph{DrawGroup}. Since every entity in \emph{DrawGroup} must participate
in the \emph{Representative} relationship exactly once, \emph{student\_id} is a
foreign key to the \emph{Student} table in each \emph{DrawGroup} tuple. The
above schema is in Third Normal Form. \\

Now, consider a parallel set of entity sets \emph{Room} and \emph{Collection}.

\begin{description}
  \item Room(
        \ul{\emph{dorm\_name}: \texttt{string},
        \emph{room\_num}: \texttt{string}},
        \emph{capacity}: \texttt{integer},
        \emph{collection\_id}: \texttt{integer})

  \item Collection(
        \ul{\emph{collection\_id}: \texttt{integer}},
        \emph{suite\_num}: \texttt{integer})
\end{description}

Here there are total participation and key constraints on \emph{Room} with
respect to the \emph{Belongs\_To} relationship.  Since each and every entity in
\emph{Room} is in \emph{Collection} exactly once, the \emph{collection\_id} can
be an attribute in \emph{Room}. It is a foreign key to the
\emph{Collection} table. Here, Third Normal Form is maintained.

Consider the tables representing the \emph{Request} and \emph{Occupy}
relationships. The former is a straightforward translation. However, the latter
is not as intuitive. We have key constraints on the attributes listed below to
ensure that a \emph{DrawGroup} can only occupy at most a single collection per
semester in an academic year. We have also implemented business logic checks to
manage the different condition combination cases.

\begin{description}
  \item Request(
        \ul{\emph{draw\_group\_id}: \texttt{integer},
        \emph{collection\_id}: \texttt{integer}},
        \emph{absolute\_rank}: \texttt{real})

  \item Occupy(
        \ul{\emph{draw\_group\_id}: \texttt{integer},
        \emph{academic\_year}: \texttt{integer},
        \emph{in\_fall?}: \texttt{boolean},
        \emph{in\_spring?}: \texttt{boolean}},
        \emph{collection\_id}: \texttt{integer})

\end{description}



