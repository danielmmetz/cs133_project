The result of the translation of the ER diagram to the schema is presented in
the next section, as well as the explanation of any non-obvious choices. First,
consider the \emph{Student} and \emph{Group} entity sets and their mutual
relationships.

\begin{description}
  \item Student(
        \ul{\emph{sid}: \texttt{integer}},
        \emph{drawNum}: \texttt{integer},
        \emph{name}: \texttt{string},
        \emph{gradYear}: \texttt{integer})

  \item Member(
        \ul{\emph{sid}: \texttt{integer},
        \emph{gid}: \texttt{integer}})

  \item DrawGroup(
        \ul{\emph{gid}: \texttt{integer}},
        \emph{drawNum}: \texttt{integer},
        \emph{repid}: \texttt{integer},
        \emph{forSuite}: \texttt{boolean})
\end{description}

The above schema mostly sticks to the obvious translation.  However, we have key
and total participation constraints on the \emph{Representative}
relationship between \emph{Student} and \emph{Group}. Since every entity in
\emph {Group} must participate in the \emph{Representative} relationship exactly
once, \emph{repid} is a foreign key to the \emph{Student} table in each
\emph{Group} tuple. The above schema is in Third Normal Form.

Now, consider a parallel set of entity sets \emph{Room} and \emph{Collection}.

\begin{description}
  \item Room(
        \ul{\emph{dormName}: \texttt{string},
        \emph{roomNumber}: \texttt{string}},
        \emph{capacity}: \texttt{integer},
        \emph{cid}: \texttt{integer})

  \item Collection(
        \ul{\emph{cid}: \texttt{integer}},
        \emph{suiteNum}: \texttt{integer})
\end{description}

Here there are total participation and key constraints on \emph{Room} with
respect to the \emph{In\_A} relationship.  Since each and every entity in
\emph{Room} is in \emph{Collection} exactly once, the \emph{cid}, or collection
ID, can be an attribute in \emph{Room}. It is a foreign key to the
\emph{Collection} table. Here, Third Normal Form is maintained.

Now consider the tables representing the \emph{Request} and \emph{Occupy}
relationships. Since there are no constraints in the ER Diagram, these are
straightforward translations.

\begin{description}
  \item Request(
        \ul{\emph{gid}: \texttt{integer},
        \emph{cid}: \texttt{integer}},
        \emph{rankAbsolute}: \texttt{real})

  \item Occupy(
        \ul{\emph{sid}: \texttt{integer},
        \emph{dormName}: string,
        \emph{roomNum}: \texttt{string}},
        \emph{academicYear}: \texttt{integer},
        \emph{inFall}: \texttt{boolean},
        \emph{inSpring}: \texttt{boolean})
\end{description}



