\noindent Below are the key aspects of the business logic that are integral to
the functionality of the Room Draw application.

\begin{itemize}
\item A student who is a representative of a group makes the request on behalf
of the group. The representative of a group is by default the student in the
group with the highest \(drawNum\).

\item On creation of a group, the \(G.drawNum\) is calculated. The manner by
which it is calculated depends on the type of group and is described as follows:

\[
    G.drawNum =
        \begin{cases}
        \text{mean} \del{S_1, \ldots, S_r} & \quad \mbox{if suite} \\
        \min \del{S_1, \ldots, S_r} & \quad \mbox{otherwise}
        \end{cases}
\]

\item For a tuple \(o \in Occupy\), \(o.inFall\) is true or \(o.inSpring\) is true.

\item Queries for Suites are a search for Collections with non-NULL \(suiteNum\)
attributes.

\item The number of students in any group must be between 1 and 6 (inclusive).
If that group is also a suite, it must have at least 3. A student can be in any
number of groups, provided that only one is a suite.

\item The attribute \(absRank\) in the \(Request\) relation is a \texttt{real} to
facilitate the ordering of requests.  We set the maximum number of requests to
be 1000, then the first three decimal places specify the student's ordered
preference. In other words, if a representative student \(s\) specifies an
ordering \(1,2,\ldots, n\), then the requests \(r_1,r_2, \ldots, r_n \in
Request\) are created such that \[r_i.absRank = s.drawNum + i/1000.0\]

\item The minimum \(drawNum\) for students is \num{10000}. This allows for
friendship suites to be given priority. On committing the requests \(r_1\ldots
r_n\), if \(r_i.cid\) corresponds to a friendship suite, \(r_i.drawNum\) is
automatically reduced by this minimum \(drawNum\), \num{10000}. This gives
requests for a friendship suite preferred \(absRank\) values.
\end{itemize}

