The automated room draw task (AutoDraw) assigns draw groups to collections in
order of absolute ranks. At the end of the task, it is possible that a user will
not have been drawn into a room. Such users must then update their preferences
and the task must be rerun for these individuals. This \texttt{rake} task must
be run on the production database, hosted by Heroku, by calling \texttt{heroku
run rake draw:auto[year]} where \texttt{year} is the fall academic year for
which users are drawing.

In order to perform the task, admin privileges on the Heroku app are required.
The admin should install Heroku toolbelt (\url{https://toolbelt.heroku.com}) and
navigate to a local repository of the project which can be found at
\url{http://github.com/danielmmetz/cs133_project}. Here, use the command
\texttt{heroku login} and follow the output instructions to finish set up. Once
this is completed, the aforementioned \texttt{rake} task may be run.