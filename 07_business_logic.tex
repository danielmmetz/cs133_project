\noindent Below are the key aspects of the business logic that are integral to
the functionality of the Room Draw application.

\begin{itemize}
\item A student who is a representative of a group makes the request on behalf
of the group. The representative of a group is the student in the group with the
lowest \emph{draw\_num}.

\item On creation of a group \(G\), the \(G.draw\_num\) is calculated. The
manner by which it is calculated depends on the type of group and is described
as follows:

\[
    G.drawNum =
        \begin{cases}
        \text{mean} \del{S_1, \ldots, S_r} & \quad \mbox{if suite \(3 \leq r \leq 6\)} \\
        \min \del{S_1, \ldots, S_r} & \quad \mbox{otherwise}
        \end{cases}
\]

\item For a tuple \(o \in Occupy\), \(o.in\_fall?\) is true or \(o.in\_spring?\) is true.

\item The number of students in any group must be between 1 and 6 (inclusive). A
group is a suite if it contains between 3 and 6 members. A student can be in up
to 50 draw groups.

\item The attribute \(absolute\_rank\) in the \(Request\) relation is a
\texttt{real} to facilitate the ordering of preferences.  We set the maximum
number of preferences to be \num{10000}, then the numbers after the decimal
specify the group's ordered preference. In other words, if a representative
student \(s\) specifies an ordering \(1,2,\ldots, n\), then the preferences
\(r_1,r_2, \ldots, r_n \in Request\) are created such that \[r_i.absolute\_rank
= s.draw\_num + \sfrac{i}{10000.0}\]

\item The minimum \(draw\_num\) for students is \num{10000}. This allows for
friendship suites to be given priority. On committing the requests \(r_1,
\ldots, r_n\), if \(r_i.collection\_id\) corresponds to a friendship suite,
\(r_i.draw\_num\) is automatically reduced by this minimum \(draw\_num\),
\num{10000}. This prioritizes friendship groups.
\end{itemize}

